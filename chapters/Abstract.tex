\chapter*{Abstract}

The universe consists of matter (protons, neutrons, etc.) as well as antimatter (antiprotons, antineutrons, etc.). If a particle collides with its respective antiparticle, these matter--antimatter particles are transformed into new force--carrier particles (e.g. photons, gluons, etc.). Therefore, if we assume, that the big bang created both, matter and antimatter equally, all matter and antimatter in the universe would have be transformed to force--carrier particles, whereby we should not exist. However, our existence implies that the amount of matter has to be higher than the amount of antimatter--we would not exist otherwise. This matter--antimatter asymmetry is one of the biggest unsolved problems in physics.\\

To solve this problem, Andrei Sakharov proposed in 1967 a set of three necessary conditions, which lead, if satisfied simultaneously, to an explanation of this asymmetry at an early stage of the universe. The third of them is the violation of the charge symmetry (C) and the charge-parity symmetry (CP), whereas C violation is already known. The CP violation could be verified by the non-zero neutron electric dipole moment (nEDM) experiment, taking place at the Garching Research Centre.\\

In the nEDM experiment, so--called ultra--cold neutrons (UCN) are used, which can be stored in traps made from certain materials because of their small kinetic energy. After their generation, the UCNs are guided into the nEDM measurement chamber. The measurement uses Ramsey's method of separated oscillatory fields applied to UCN in a double--chamber layout, combined with different means of magnetometry. So, if the experiment shows a non-zero value of the neutron's electric dipole moment, this would manifest the yet unknown time reversal symmetry (T) violation. Assuming the conservation of the combined charge, parity and time symmetry (CPT), this would imply the violation of the CP symmetry, which would verify Sakharov's third condition.\\

The large amount of data provided by the experiment's setup has to be handled by an appropriate Data Acquisition (DAQ) system, whereas we distinguish between slow, permanent submitted data (``slow control'') and fast acquired data (``fast control''). The slow control system collects a lot of parameters around the experiment, like field values, vacuum quality and temperature, but can also control certain devices of the experiment. In contrast to the slow control system, the fast control system is responsible for data, which has to be processed in real time to regulate the experiment's parameters.\\

The basic structure of the overall DAQ system consists of hardware devices, a database and view--controlling programs. Principally all of the used measurement devices need a PC or a small controlling device. The reading out and transfer of the data is done with software like ORCA, LabView or certain Python scripts. They are also responsible for controlling the devices via parameters obtained from a database, whereas for the project the database CouchDB should be used. CouchDB allows the implementation of certain applications called CouchApps, which are hosted by the database system and can be used to access the stored data. These CouchApps are web based applications, mainly implemented in JavaScript, and therefore platform independent, and furthermore accessible via a web browser from everywhere. 